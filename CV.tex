\documentclass[10pt,a4paper,ragged2e,withhyper]{altacv}

\geometry{left=1.25cm,right=1.25cm,top=1.5cm,bottom=1.5cm,columnsep=1.2cm}

\usepackage{paracol}

\definecolor{SlateGrey}{HTML}{2E2E2E}
\definecolor{LightGrey}{HTML}{666666}
\definecolor{DarkPastelRed}{HTML}{450808}
\definecolor{PastelRed}{HTML}{8F0D0D}
\definecolor{GoldenEarth}{HTML}{E7D192}
\colorlet{name}{black}
\colorlet{tagline}{PastelRed}
\colorlet{heading}{DarkPastelRed}
\colorlet{headingrule}{GoldenEarth}
\colorlet{subheading}{PastelRed}
\colorlet{accent}{PastelRed}
\colorlet{emphasis}{SlateGrey}
\colorlet{body}{LightGrey}

\renewcommand{\namefont}{\Huge\rmfamily\bfseries}
\renewcommand{\personalinfofont}{\footnotesize}
\renewcommand{\cvsectionfont}{\LARGE\rmfamily\bfseries}
\renewcommand{\cvsubsectionfont}{\large\bfseries}

\renewcommand{\cvItemMarker}{{\small\textbullet}}
\renewcommand{\cvRatingMarker}{\faCircle}

\begin{document}
\name{Alessandro Celadon}
\tagline{Full Stack Developer}
\photoR{2.5cm}{propic.jpeg}
\personalinfo{
  \email{aleceladon@gmail.com}
  \phone{347-5576930}
  \location{Italia}
  \homepage{cela24k.github.io}
  \linkedin{linkedin.com/in/alessandro-celadon-6b6a9329b/}
  \github{Cela24k}
}

\makecvheader
\columnratio{0.6}
\begin{paracol}{2}

\cvsection{Esperienza}

\cvevent{Tirocinante Sviluppatore Software}{Autoware Srl}{Luglio-Settembre 2024}{Italia}
\begin{itemize}
  \item Progettazione di una piattaforma di scambio dati configurabile per sincronizzare dati da una piattaforma on-premise a un database locale.
  \item Tecnologie: C\#, PostgreSQL, REST API.
\end{itemize}

\cvsection{Progetti}

\cvevent{Portale di gestione progetti di ricerca universitario}{Flask + SQLAlchemy ORM}{}{}
\begin{itemize}
  \item Creazione di un portale per la gestione di progetti di ricerca.
\end{itemize}

\cvevent{Showcase prodotti 3D}{React, HTML, CSS, ThreeJS}{}{}
\begin{itemize}
  \item Sviluppo di un sito vetrina per prodotti 3D interattivi.
\end{itemize}

\cvevent{Battaglia Navale Full Stack}{MEAN Stack}{}{}
\begin{itemize}
  \item Implementazione di un gioco di battaglia navale con Angular, Express.js, MongoDB e Node.js.
\end{itemize}

\cvsection{Competenze}

\cvtag{Sviluppo Web}
\cvtag{REST API}
\cvtag{Node.js, JavaScript, TypeScript}
\cvtag{SQL (PostgreSQL), NoSQL (MongoDB)}
\cvtag{Linux, Windows}
\cvtag{Flask, SQLAlchemy ORM}
\cvtag{Stack MERN e MEAN}


\cvsection{Linguaggi di Programmazione}

\cvskill{Java}{5}
\divider
\cvskill{C, C++}{4.5}
\divider
\cvskill{JavaScript / TypeScript / Nodejs}{4.5}
\divider
\cvskill{Python}{4}
\divider
\cvskill{C\#}{3.5}
\divider

\switchcolumn

\cvsection{Formazione}

\cvevent{Studente di Informatica}{Universit\`a Ca' Foscari Venezia}{2019 - 2025}{}
\begin{itemize}
  \item Studio approfondito di sviluppo software, sicurezza informatica e database.
  \item Autodidatta in HTML, CSS, ThreeJS, Sicurezza Informatica.
\end{itemize}

\cvevent{Diploma Scientifico}{Liceo G.B. Quadri}{2014 - 2019}{}

\cvsection{Certificazioni}

\cvachievement{\faCertificate}{Certificazione TOEFL B2}{Competenze linguistiche in inglese avanzate.}

\cvsection{Referenze}

\cvref{Prof. Cortesi}{Universit\`a di Venezia}{cortesi@unive.it}{}
\divider
\cvref{Diego Maron}{Autoware Srl}{diego.maron@autoware.it}{}

\end{paracol}
\end{document}
